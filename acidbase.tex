\chapter{Acids \& Bases}\label{acidbase}
Liquid water partially dissociates into \ce{H+} and \ce{OH-} ions:
\[\ce{H2O(l) <=>[K_w] H+(aq) + OH-(aq)}\]
At equilibrium, the molar concentrations (denoted by square brackets) of these ions are related to the auto-ionization constant \ce{K_w}:
\begin{equation}
    \ce{K_w = [H+][OH^-]}=\qty{e-14}{M\squared}\label{kw}
\end{equation}
In neutral water, the concentration of each species is balanced:
\begin{equation}
    \ce{[H+] = [OH^-]}=10^{-7}M\label{neut}
\end{equation}
Acids and bases disrupt \eqref{neut} such that \ce{[H+] \neq [OH^-]} by donating or accepting protons, respectively. However, they do not ultimately alter the equilibrium relation \eqref{kw}. This is important because it means \ce{[H+]} and \ce{[OH-]} are totally dependent on one another; the entire system can be characterized by only one variable. pH is the most commonly used parameter:
\[\ce{pH=-\log[H+]}\]
\section{Conjugate Pairs}
Acids and bases come in \textit{conjugate pairs.} After an acid donates a proton, the remaining species is a conjugate base. After a base accepts a proton, it becomes a conjugate acid.
\section{Strong \& Weak Species}
Strong acids and bases participate in proton exchange more readily than weak ones. The strength of conjugate pairs is inverse, i.e., a strong acid has a weak conjugate base and vice versa.
For example, nitric acid is a strong acid which readily donates its proton into solution, setting up an unbalanced equilibrium:
\[\ce{HNO3 <=>> H+ + NO3^-}\]
Since nitric acid is so effective at donating its proton, the conjugate base \ce{NO3^-} is very weak and does not act as an effective base. By contrast, weak species like carbonic acid set up more balanced equilibria:
\[\ce{H2CO3 <=> H+ + HCO3-}\]
The conjugate base \ce{HCO3-} is strong enough participate in proton exchange and thus qualifies as a base. The distinction between strong and weak species allows us to proceed with a definition of Alkalinity.