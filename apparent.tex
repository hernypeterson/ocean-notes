\newcomand{\af}{\ensuremath{\Epsilon \textbf{F}_a}}
\chapter{The Coriolis Effect}
\section{Motivation}
Although we do not feel the effects of Earth's rotation, our experiences with rotation on a more familiar scale implies some change in the in particular, the flow of water and air. 

Our goal is to derive a concise set of physical laws which allow us to easily translate between an inertial reference (where Newtonian mechanics is valid) and a non-inertial reference, which is more convenient for measurement and description. We will do this first in the most general case, and then take the special case of the rotating Earth.

The conceptual framework is as follows
\begin{enumerate}
\item A position in space may be described using any number of coordinate systems; the position of the origin and orientation of axes will give different numerical values to the same position 
\end{enumerate}



\section{Derivation of Apparent Forces}
Newtons second law \cref{newton2} applies only to inertial frames of reference. In other words, the equation will accurately describe and predict motion if the observer is not accelerating under the influence of a force.