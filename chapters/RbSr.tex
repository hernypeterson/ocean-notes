\documentclass[12pt,addpoints]{exam}
\usepackage[utf8]{inputenc}
\usepackage{amsmath}
\usepackage[version=4]{mhchem}
\usepackage{siunitx}
\usepackage{microtype}
\usepackage{libertine}
\usepackage{parskip}
\usepackage[libertine]{newtxmath}
\usepackage{isotope}
\bonuspointpoints{bonus point}{bonus points}
\usepackage{mhchem}
\title{Rb/Sr Dating}
\author{GEOL 189C}
\begin{document}
\maketitle
\isotope[87]{Rb} decays to \isotope[87]{Sr} with decay constant $\lambda=\qty{1.393e-11}{y^{-1}}$. So the amount of \isotope[87]{Rb} remaining at time $t$ is related to the initial amount $\left(\isotope[87]{Rb}_0\right)$ as follows:
\begin{equation}
    \isotope[87]{Rb}(t)=\isotope[87]{Rb}_0\cdot e^{-\lambda t}\label{decay}
\end{equation}
\isotope[87]{Sr} accumulates at the same rate that \isotope[87]{Rb} decays, so the accumulated amount of \isotope[87]{Sr} is the amount of decayed \isotope[87]{Rb}, the difference $\isotope[87]{Rb}_0-\isotope[87]{Rb}(t)$. Including initial $\isotope[87]{Sr}_0$:
\begin{equation}
    \isotope[87]{Sr}(t)=\isotope[87]{Sr}_0+\isotope[87]{Rb}_0-\isotope[87]{Rb}(t)
\end{equation}
\isotope[86]{Sr} is a stable, non-radiogenic isotope used to normalize each measurement. For example:
\begin{equation*}
    \frac{\isotope[87]{Rb}}{\isotope[86]{Sr}}(t)=\frac{\isotope[87]{Rb}_0}{\isotope[86]{Sr}}\cdot e^{-\lambda t}\tag{\ref{decay}}
\end{equation*}

\hrule

The following data were sampled from a single outcrop which we assume formed simultaneously:
\begin{center}
    \begin{tabular}{ccc}
    \isotope[87]{Rb}/\isotope[86]{Sr} & \isotope[87]{Sr}/\isotope[86]{Sr}\\
    \hline
    3.01 & 0.70995\\
    2.00 & 0.7082\\
    2.63 & 0.7092\\
    1.92 & 0.70775
\end{tabular}
\end{center}
\begin{questions}
    \question[2] How should each ratio to change over time, especially \emph{relative to one another?}
    \question[2] How could you determine the initial ratios of each sample?
    \question[2] How could you determine the time elapsed since this initial state?
    \question[2] What is your best estimate for the age of the outcrop?
    \question[2] What conditions must be true for your interpretation to be valid?
\end{questions}
\end{document}
