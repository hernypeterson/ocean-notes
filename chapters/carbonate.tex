\chapter{Inorganic Carbon Cycle}\label{carbonate}
\section{Equilibria}
The atmospheric concentration of carbon dioxide (\ce{pCO2}) is a primary factor influencing global climate today and through deep time. The oceanic inorganic carbon reservoir is over 50 times that of the atmosphere, so atmosphere-ocean exchange drives \ce{pCO2} over millennial timescales. The goal of this section is to derive a quantitative model which relates \ce{pCO2} to the measurable ocean carbonate system.

When \ce{CO2} dissolves, it reacts with water to form carbonic acid (\ce{H2CO3^*})\footnote{It is difficult to distinguish between dissolved carbon dioxide and carbonic acid, so they are grouped as a single species}, bicarbonate (\ce{HCO3-}), and carbonate (\ce{CO3^2-}). These reactions are reversible, so the following equilibrium is established:
\begin{multline}
    \ce{CO2 (g) + H2O (l) <=>[K0] H2CO3^* (aq) \\ <=>[K1] H+ (aq) + HCO3- (aq) \\ <=>[K2] 2H+ (aq) + CO3^2- (aq)}\label{sys}
\end{multline}
\ce{K_{0,1,2}} are measurable concentration ratios of reactants by products at equilibrium:
\begin{gather}
    \ce{K0 = [H_2CO_3^{*}]/pCO_2}\label{K0}\\
    \ce{K1 = [H^+][HCO_3^-]/[H_2CO_3^*]}\label{K1}\\
    \ce{K2 = [H^+][CO_3^{2+}]/[HCO_3^-]}\label{K2}
\end{gather}
Dissolved Inorganic Carbon (DIC) and Alkanility (Alk) are also measurable:
\begin{multline}
    \ce{DIC = \underset{0.5\%}{[H_2CO_3^*]} + \underset{88.6\%}{[HCO_3^-]} + \underset{10.9\%}{[CO_3^{2-}]} \\ \approx [HCO3^-] + [CO3^2-]}\label{DICappx}
\end{multline}
Alk is the charge-balanced\footnote{Concentrations of species which exchange more than one proton are scaled accordingly.} excess of bases in solution, primarily carbonic conjugate bases:
\begin{multline}
     \ce{Alk = \underset{76.8\%}{[HCO_3^-]} + \underset{18.7\%}{2[CO_3^{2-}]} + \\ \underset{4.2\%}{[B(OH)_4^-]} + \underset{0.2\%}{(\Sigma[B] - \Sigma[A])} \\
     \approx [HCO3^-] + 2[CO3^2-]\label{alkc}}
\end{multline}
\section{Derivation}
Rearranging equilibrium constant expressions gives:
\begin{gather}
    \ce{[H+] = K2.[HCO3^-]/[CO3^2-]}\tag{\ref{K2}}\\
    \ce{[H2CO3^*] = [H+][HCO3^-]/K1}\tag{\ref{K1}}\\
    \ce{pCO2 = [H_2CO_3^{*}]/K_0}\tag{\ref{K0}}
\end{gather}
Sequential substitution and simplification gives:
\begin{equation}
    \ce{pCO2 = \frac{K2}{K0.K1}\cdot\frac{[HCO3^-]^2}{[CO3^{2-}]}}\label{CO2K}
\end{equation}
Subtracting~\eqref{alkc} from \eqref{DICappx} and substituting the result back into \eqref{DICappx} gives:
\begin{gather}
    \ce{[CO3^2-] \approx Alk - DIC}\\
    \ce{[HCO3^2-] \approx $2$DIC - Alk}
\end{gather}
Finally, substituting back into~\eqref{CO2K} gives:
\begin{equation}
\ce{pCO2 \approx \dfrac{\ce{K2}}{\ce{K0\cdot K1}}\cdot \dfrac{(\ce{$2$DIC - Alk})^2}{\ce{Alk - DIC}}}
\end{equation}
\section{Calcium}
The complexity of the carbonate system leads to some counter-intuitive dynamics. One of these is the observation that an increase in \ce{[H2CO3^*]} has the effect of \textit{lowering} \ce{[CO3^2-]}. To understand why, recall from \eqref{sys}:
\[\ce{H2CO3}^{*}\ce{<=>[K1] H+(aq) + HCO3^-(aq)}\]
By Le Chatelier's principle, increasing the concentration of the reactant \ce{H2CO3^*} will shift the reaction to the right, and thus more \ce{H+} and \ce{HCO3-} will be produced in equal measure, say by amount a. Next, consider how this increase affects the balance of the next equilibrium \eqref{sys}:
\begin{equation}
    \ce{HCO3-(aq) <=>[K2] H+(aq) + CO3^2-(aq)}
\end{equation}
Recall that the equilibrium concentrations of the products in this reaction are much lower than the product. Thus, the same increase + a to both sides will have a greater effect on the products, shifting the equation to the left (converting \ce{CO3^2-} to \ce{HCO3-}). With the addition of carbonic acid, carbonate alkalinity is consumed. Or, in terms of the equilibrium constant expression:
\begin{equation}
    \ce{K2 = \dfrac{[H^+][CO_3^{2+}]}{[HCO_3^-]}}\tag{\ref{K2}}
\end{equation}
Solving for \ce{[CO3^2-]}:
\[\ce{[CO3^2-] \propto \dfrac{[HCO3-]}{[H+]}}\gg1\]
Clearly, as a increases with the addition of \ce{H2CO3^*}, \ce{[CO3^2-]} will decrease.
The concentration of carbonate is of key concern in another geochemical system, the biological calcium carbonate pump.

Calcium carbonate forms two minerals, calcite and aragonite, which many ocean organisms use to build shells and exoskeletons.
\[\ce{CaCO3(s) <=>[K_{sol}] Ca^2+(aq) + CO3^2-(aq)}\]
\[\ce{K_{sol}=[Ca^2+][CO3^2-]}\]
\ce{[Ca^2+]} = \qty{10}{\mmol\per\kg}
\\\ce{[CO3^2-]} $\approx$ \qtyrange[range-units = single]{40}{200}{\umol\per\kg}